\documentclass[twocolumn]{article}
\usepackage[portrait,margin=2cm]{geometry}

\title{AI 1103 Assignment-2}
\author{Shantanu Pandey\\ CS20BTECH11046}
\date{}
\usepackage{amsmath}
\newtheorem{theorem}{Theorem}[section]
\newtheorem{proof}[theorem]{Proof}
\usepackage{amssymb}
\usepackage{amsfonts}
\usepackage{nopageno}
% \usepackage[margin=1in]{geometry}
\usepackage{graphicx}
\usepackage{float}
\usepackage{multicol}
\usepackage{enumitem}
\usepackage{hyperref}
%\setlength{\columnsep}{0.5cm}
%\setlength{\parindent}{0em}
\usepackage{color}
\usepackage{comment}
%\setlength{\columnseprule}{1pt}
%\def\columnseprulecolor{\color{black}}

\providecommand{\mbf}{\mathbf}
\providecommand{\pr}[1]{\ensuremath{\Pr\left(#1\right)}}
\providecommand{\qfunc}[1]{\ensuremath{Q\left(#1\right)}}
\providecommand{\sbrak}[1]{\ensuremath{{}\left[#1\right]}}
\providecommand{\lsbrak}[1]{\ensuremath{{}\left[#1\right.}}
\providecommand{\rsbrak}[1]{\ensuremath{{}\left.#1\right]}}
\providecommand{\brak}[1]{\ensuremath{\left(#1\right)}}
\providecommand{\lbrak}[1]{\ensuremath{\left(#1\right.}}
\providecommand{\rbrak}[1]{\ensuremath{\left.#1\right)}}
\providecommand{\cbrak}[1]{\ensuremath{\left\{#1\right\}}}
\providecommand{\lcbrak}[1]{\ensuremath{\left\{#1\right.}}
\providecommand{\rcbrak}[1]{\ensuremath{\left.#1\right\}}}


\newcommand*{\permcomb}[4][0mu]{{{}^{#3}\mkern#1#2_{#4}}}
\newcommand*{\perm}[1][-3mu]{\permcomb[#1]{P}}
\newcommand*{\comb}[1][-1mu]{\permcomb[#1]{C}}

\begin{document}
\newtheorem{theorem}{Theorem}[section]
\newtheorem{problem}{Problem}
\newtheorem{proposition}{Proposition}[section]
\newtheorem{lemma}{Lemma}[section]
\newtheorem{corollary}[theorem]{Corollary}
\newtheorem{example}{Example}[section]
\newtheorem{definition}[problem]{Definition}

\newcommand{\BEQA}{\begin{eqnarray}}
\newcommand{\EEQA}{\end{eqnarray}}
\newcommand{\define}{\stackrel{\triangle}{=}}
\newcommand{\Int}{\int\limits}
\maketitle
\noindent

\begin{figure} [h]
    \includegraphics[width = 0.6\columnwidth]{Logo.png}
\end{figure}
\vspace{0.3cm}

Download all latex-tikz codes from  

\vspace{0.3cm}  
    
\fbox{%
    \parbox{0.45\textwidth}{%
        \url{https://github.com/Shantanu467/AI1103/blob/main/Assignment_2/Assignment_2.tex}
    }%
    }
   
\vspace{0.5cm}
\section*{Problem}
\textbf{GATE EC: Question-74 } \\
 Let $X_1$ be an exponential random variable with
mean $1$ and $X_2$ a gamma random variable with
mean $2$ and variance $2$. If $X_1$ and $X_2$ are
independently distributed, then $\mathrm{P}\brak{X_1 < X_2}$
is equal to.....

\section*{Solution}
\begin{definition}
The probability density function (pdf) of an \textbf{ exponential distribution} is
\begin{align}
    f\brak{x;\lambda} &=
    \begin{cases}
    \lambda \mathrm{e}^{-\lambda x} , & x  \geq 0 \\
     0, &  x < 0
    \end{cases}
\end{align}
Here $\lambda > 0$ is the parameter of the distribution, often called the rate parameter.
\end{definition}
\begin{lemma}
The \textbf{mean} of an exponentially distributed random variable $X$ with rate parameter $\lambda$ is given by
\begin{align}
    \mathrm{E}\brak{X} =\frac{1}{\lambda}
\end{align}
\end{lemma}
\begin{proof}
\begin{align}
       \mathrm{E}\brak{X} &= \int_{-\infty}^{+\infty} xf_X\brak{x} \,\mathrm{d}x \\
       \mu &= \int_{-\infty}^{0} 0 \times x \, \mathrm{d}x + \int_{0}^{+\infty} \lambda\mathrm{e}^{-\lambda x} \times x \, \mathrm{d}x \\
       &= 0 + \brak{- \frac{\brak{\lambda x + 1 } \mathbf{e}^{-\lambda x}} {\lambda}}\Bigr|_{0}^{+\infty } \\
       &= 0 - \brak{- \frac{\brak{0+1}\mathrm{e}^{0} }{\lambda}} \\
     \mu  &= \frac{1}{\lambda} \label{Expmu}
\end{align}
\end{proof}
Now, using \eqref{Expmu} and $\mu = 1$ \brak{Given}, we get; 
 \begin{align}
    \mu &= \frac{1}{\lambda} = 1 \\
\implies    \lambda &= 1 \label{explambda}
\end{align}

So, \begin{align}
 X_1  \sim  f\brak{x;1} = \mathrm{e}^{-x} \label{FX1}
\end{align}


\begin{definition}
The probability density function (pdf) of an \textbf{ gamma distribution} is
\begin{align}
f\brak{x;\lambda} &=
    \begin{cases}
    \dfrac{a^{\lambda}}{\Gamma{\lambda}\ } e^{ax}x^{\lambda-1} , & x > 0 \\
     0, &  x \leq 0
    \end{cases}
        \end{align}
Here, $\lambda > 0 $, $a>0$ and
\begin{align}
    \Gamma{n} &= a^{n} \int_{0}^{\infty} \mathrm{e}^{ax} x^{n -1} \, \mathrm{d}x \label{Gamma} \\
 \Gamma{n} &= \brak{n-1}! \label{GammaFac}   \end{align} 
\end{definition}
\begin{lemma}
The \textbf{mean} and \textbf{variance} of an exponentially distributed random variable $X$ with $\lambda$ and $a$ is given by
\begin{align}
    \mu &= \frac{\lambda}{a} \\
    \sigma^2 &= \frac{\lambda}{a^2}
\end{align}
\end{lemma}
\begin{proof}
\begin{align}
 \mathrm{E}\brak{X} &= \int_{-\infty}^{+\infty} xf_X\brak{x} \,\mathrm{d}x \\
 \mu &= 0 + \int_{0}^{\infty} x \times \dfrac{a^{\lambda}}{\Gamma{\lambda}\ } e^{ax}x^{\lambda-1} \, \mathrm{d}x \\
 &= 0 + \frac{ a^{\lambda}}{\Gamma \lambda} \int_0^\infty e^{ax}x^{\lambda} \, \mathrm{d}x  \label{meanint} 
 \end{align}
Now by using,  n = $ \lambda +1 $ in \eqref{Gamma} we get;
 \begin{align} 
  \Gamma \brak{ \lambda +1} &= a^{\lambda +1} \int_0^\infty e^{ax}x^{\lambda} \, \mathrm{d}x  \label{Gamma1}
  \end{align}
  Using \eqref{Gamma1} in \eqref{meanint}, we get;
  \begin{align}
 \mu &= \frac{a^\lambda}{\Gamma \lambda} \times \frac{\Gamma \brak{\lambda +1}}{ a^{\lambda +1} } \\
 &= \frac{1}{a} \times \frac{\Gamma \brak{\lambda+1}}{ \Gamma \lambda} 
 \end{align}
 Using  \eqref{GammaFac} to find $\Gamma \lambda$
 \begin{align}
 \mu &= \frac{1}{a} \times \frac{\lambda !}{\brak{\lambda -1 }!} \\
 \mu &= \frac{\lambda}{a} \label{GammaMean}
 \end{align}
Similarly we can find E\brak{X^2} .
\begin{align}
    \mathrm{E}\brak{X} &= \frac{\lambda \brak{\lambda +1}}{ a^2} \\
    \sigma^{2} &= \brak{\mathrm{E}\brak{X}}^2 - \mathrm{E}\brak{X^2} \\
    &=  \frac{\lambda}{a^2} \label{GammaVar}
\end{align}

\end{proof}
Now by using \eqref{GammaMean} \eqref{GammaVar} And given info; \\
\begin{align}
\mu &= \dfrac{\lambda}{a}=2 
\implies \lambda = 2a \\
\sigma^2 &=  \dfrac{\lambda}{a^2}=\dfrac{2a}{a^2}=2  \\ \implies a &= 1,\lambda=2 \\
X_2\sim G(1,2) &= \dfrac{1}{\Gamma{2}\ } e^{x}x \\
 &= e^{x}x \label{FX2}
\end{align}


Now by the help of \eqref{FX1} and \eqref{FX2}, we can solve for $\mathrm{P} \brak{X_1 < X_2} $ .

\begin{align}
  \mathrm{P}(X_1<X_2) &= \mathrm{P}\brak{X_1 < X_2 \mid X_1 = X_2} \\
  &= \int \limits_{0}^{\infty} fX_2 \brak{x_2} \times \int \limits_{0}^{x_2} fX_1 \brak{x_1} \, \mathrm{d}x_1 \, \mathrm{d}x_2 \\
  &= \int \limits_{0}^{\infty} x_2 \mathrm{e}^{-x_2} \times \int \limits_{0}^{x_2} \mathrm{e}^{-x_1} \, \mathrm{d}x_1 \, \mathrm{d}x_2 \\
  &= \int \limits_{0}^{\infty} x_2 \mathrm{e}^{-x_2} \times \brak{1-\mathrm{e}^{-x_2} } \, \mathrm{d}x_2 
\end{align}
Upon solving the definite integral, We get :
\begin{align}
   \mathrm{P}(X_1<X_2) &= \mathbf{ \frac{3}{4} } 
\end{align}
 % \end{multicols*}


\end{document}
